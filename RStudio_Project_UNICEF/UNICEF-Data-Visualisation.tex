% Options for packages loaded elsewhere
\PassOptionsToPackage{unicode}{hyperref}
\PassOptionsToPackage{hyphens}{url}
%
\documentclass[
]{article}
\usepackage{amsmath,amssymb}
\usepackage{lmodern}
\usepackage{iftex}
\ifPDFTeX
  \usepackage[T1]{fontenc}
  \usepackage[utf8]{inputenc}
  \usepackage{textcomp} % provide euro and other symbols
\else % if luatex or xetex
  \usepackage{unicode-math}
  \defaultfontfeatures{Scale=MatchLowercase}
  \defaultfontfeatures[\rmfamily]{Ligatures=TeX,Scale=1}
\fi
% Use upquote if available, for straight quotes in verbatim environments
\IfFileExists{upquote.sty}{\usepackage{upquote}}{}
\IfFileExists{microtype.sty}{% use microtype if available
  \usepackage[]{microtype}
  \UseMicrotypeSet[protrusion]{basicmath} % disable protrusion for tt fonts
}{}
\makeatletter
\@ifundefined{KOMAClassName}{% if non-KOMA class
  \IfFileExists{parskip.sty}{%
    \usepackage{parskip}
  }{% else
    \setlength{\parindent}{0pt}
    \setlength{\parskip}{6pt plus 2pt minus 1pt}}
}{% if KOMA class
  \KOMAoptions{parskip=half}}
\makeatother
\usepackage{xcolor}
\usepackage[margin=1in]{geometry}
\usepackage{longtable,booktabs,array}
\usepackage{calc} % for calculating minipage widths
% Correct order of tables after \paragraph or \subparagraph
\usepackage{etoolbox}
\makeatletter
\patchcmd\longtable{\par}{\if@noskipsec\mbox{}\fi\par}{}{}
\makeatother
% Allow footnotes in longtable head/foot
\IfFileExists{footnotehyper.sty}{\usepackage{footnotehyper}}{\usepackage{footnote}}
\makesavenoteenv{longtable}
\usepackage{graphicx}
\makeatletter
\def\maxwidth{\ifdim\Gin@nat@width>\linewidth\linewidth\else\Gin@nat@width\fi}
\def\maxheight{\ifdim\Gin@nat@height>\textheight\textheight\else\Gin@nat@height\fi}
\makeatother
% Scale images if necessary, so that they will not overflow the page
% margins by default, and it is still possible to overwrite the defaults
% using explicit options in \includegraphics[width, height, ...]{}
\setkeys{Gin}{width=\maxwidth,height=\maxheight,keepaspectratio}
% Set default figure placement to htbp
\makeatletter
\def\fps@figure{htbp}
\makeatother
\setlength{\emergencystretch}{3em} % prevent overfull lines
\providecommand{\tightlist}{%
  \setlength{\itemsep}{0pt}\setlength{\parskip}{0pt}}
\setcounter{secnumdepth}{-\maxdimen} % remove section numbering
\ifLuaTeX
  \usepackage{selnolig}  % disable illegal ligatures
\fi
\IfFileExists{bookmark.sty}{\usepackage{bookmark}}{\usepackage{hyperref}}
\IfFileExists{xurl.sty}{\usepackage{xurl}}{} % add URL line breaks if available
\urlstyle{same} % disable monospaced font for URLs
\hypersetup{
  pdftitle={IMPACTS OF MALNUTRITION},
  hidelinks,
  pdfcreator={LaTeX via pandoc}}

\title{IMPACTS OF MALNUTRITION}
\author{}
\date{\vspace{-2.5em}}

\begin{document}
\maketitle

\begin{longtable}[]{@{}
  >{\raggedright\arraybackslash}p{(\columnwidth - 8\tabcolsep) * \real{0.1354}}
  >{\raggedright\arraybackslash}p{(\columnwidth - 8\tabcolsep) * \real{0.1250}}
  >{\raggedright\arraybackslash}p{(\columnwidth - 8\tabcolsep) * \real{0.4167}}
  >{\raggedright\arraybackslash}p{(\columnwidth - 8\tabcolsep) * \real{0.1979}}
  >{\raggedright\arraybackslash}p{(\columnwidth - 8\tabcolsep) * \real{0.1250}}@{}}
\toprule()
\begin{minipage}[b]{\linewidth}\raggedright
Name
\end{minipage} & \begin{minipage}[b]{\linewidth}\raggedright
Roll Number
\end{minipage} & \begin{minipage}[b]{\linewidth}\raggedright
Assignment
\end{minipage} & \begin{minipage}[b]{\linewidth}\raggedright
Date of Submission
\end{minipage} & \begin{minipage}[b]{\linewidth}\raggedright
Module Code
\end{minipage} \\
\midrule()
\endhead
Rishov Ghosh & 22267770 & Data Visualisation with R (UNICEF DATA) &
30.4.2023 & MT5000 \\
\bottomrule()
\end{longtable}

\includegraphics[width=0.25\linewidth]{UNICEF_Symbol}
\includegraphics[width=0.25\linewidth]{DCU}

\textbf{Introduction}

Malnutrition is a significant worldwide health issue, impacting a vast
number of individuals, particularly those who are young children aged
five and below. Malnutrition has been found to have adverse effects not
only on physical growth and cognitive development but also on the
likelihood of illness, disability, life expectancy and mortality.
Comprehending the patterns and trends of malnutrition and its
relationship with mortality, and life expectancy rates across diverse
regions is imperative for formulating efficacious interventions and
policies to address this pressing concern.

This study presents a thorough examination of the correlation between
malnutrition, mortality rates, and life expectancy utilizing data
sourced from UNICEF. Our objective is to utilize diverse data
visualization methods to emphasize the relationship between malnutrition
and other said indicators among distinct continents and nations. A range
of visual representations, such as a global map, bar graph, scatterplot
with a regression line, and time-series chart, will be utilized to
enhance comprehension of the worldwide malnutrition terrain and its
ramifications.

\textbf{World Map of Malnutrition}

The geographical map visually represents an overview of the regions with
malnutrition, with dark red being the most and white being the least
affected regions by malnutrition. It can be seen that malnutrition under
the age of 5 is the most prevalent in Africa and distributed in linear
proportion across Eastern Europe, Asia and South America.

\includegraphics{UNICEF-Data-Visualisation_files/figure-latex/worldmap-1.pdf}

\textbf{Bar Chart of Mortality Rate by Sex/Region}

The bar chart shows the mean mortality rate per 1000 individuals by sex
and region. It can be seen that malnutrition under the age of 5 is the
most prevalent in Africa and distributed in linear proportion across the
rest of the continents.

\includegraphics{UNICEF-Data-Visualisation_files/figure-latex/bar chart-1.pdf}

\textbf{Scatterplot of Life Expectancy at Birth}

The scatterplot indicates the continental distribution of life
expectancy at birth with malnutrition. It shows that life expectancy is
the lowest in Africa and highest in Europe. This implies that
malnutrition levels are inversely proportional to the life expectancy of
the regions. Since Africa shows the highest malnutrition levels overall,
life expectancy is the lowest.

\includegraphics{UNICEF-Data-Visualisation_files/figure-latex/scatterplot with regression line-1.pdf}

\textbf{Time Series of Mortality Rate}

The time series shows the trend of mortality rate since 2000. It shows a
downward trend, indicating that over the 20-year period, the mortality
rate is coming down, suggesting that malnutrition levels are coming down
with time.

\includegraphics{UNICEF-Data-Visualisation_files/figure-latex/Time Series-1.pdf}

\end{document}
